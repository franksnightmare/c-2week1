\documentclass[11pt]{article}

\usepackage{times}
\usepackage[english]{babel}

% -----------------------------------------------
% especially use this for you code
% -----------------------------------------------

\usepackage{courier}
\usepackage{listings}
\usepackage{color}
\usepackage{tabularx}
\usepackage{graphicx}

\definecolor{Gray}{gray}{0.95}

\definecolor{mygreen}{rgb}{0,0.6,0}
\definecolor{mygray}{rgb}{0.5,0.5,0.5}
\definecolor{mymauve}{rgb}{0.58,0,0.82}

\lstset{language=C++,
	basicstyle = \normalsize\ttfamily,   % the size and fonts that are used
	tabsize = 2,                    % sets default tabsize
	breaklines = true,              % sets automatic line breaking
	keywordstyle=\color{blue}\ttfamily,
	stringstyle=\color{red}\ttfamily,
	commentstyle=\color{mygreen}\ttfamily,
	numbers=left,
	keepspaces=true,
	showspaces=false,
	showstringspaces=false,
}

\begin{document}

\title{Programming in C/C++ \\
       Exercises set eight: Overloading
}
\date{\today}
\author{Christiaan Steenkist \\
Jaime Betancor Valado \\
Remco Bos \\
}

\maketitle

\section*{Exercise 1, catching and throwing references}
\subsection*{Exercise description}
There are 3 parts to this exercise:
\begin{itemize}
\item Show that exception catchers catching objects result in additional copies of thrown objects, compared to exception catchers catching references to objects.
\item Also show that when throwing objects or references copies of the (referred to) objects are thrown.
\item Also answer the question whether `throw;' results in throwing the currently available exception or a copy of that exception. 
\end{itemize}

\subsection*{Part 1, Throwing by value, catching by value}
Throwing object 'main object'by value
Caught exception by value
Hello by 'local object' (copy)  (copy) // 2 copies are found

-Throwing by value catching by reference:
\begin{lstlisting}
Throwing object 'main object' by value
Caught exception by reference
Hello by 'local object' (copy) // 1 copy is found (answered part 2)
\end{lstlisting}

\subsection*{Part 2}
The '(copy)' is appended by the copy constructor, so atleast 1 copy is made
by throwing an object.

\subsection*{Part 3}
'Throw' throws the original exception. An exception is rethrown when it is not caught yet in the present try-block level, then the exception will be retrown to a higher level until it is caught.
That means that the exception is handled and will be inactivated.

\subsection*{Code listings}
\lstinputlisting[caption = demo.h]{src/a1/demo.h}
\lstinputlisting[caption = main.cc]{src/a1/main.cc}

\section*{Exercise 3, exceptions in the \texttt{Strings} class}
Exception handling has been put into the \texttt{Strings} class.
Generally bad allocations are handled by the class itself.
The constructor can still throw bad allocation exceptions in case there is not enough memory to create a strings class.

\subsection*{Code listing}
\lstinputlisting[caption = strings.h]{src/a3/strings/strings.h}
\lstinputlisting[caption = add.cc]{src/a3/strings/add.cc}
\lstinputlisting[caption = destroy2.cc]{src/a3/strings/destroy2.cc}
\lstinputlisting[caption = enlarged.cc]{src/a3/strings/enlarged.cc}
\lstinputlisting[caption = reserve.cc]{src/a3/strings/reserve.cc}
\lstinputlisting[caption = resize.cc]{src/a3/strings/resize.cc}
\lstinputlisting[caption = storagearea.cc]{src/a3/strings/storagearea.cc}
\lstinputlisting[caption = strings1.cc]{src/a3/strings/strings1.cc}
\lstinputlisting[caption = strings2.cc]{src/a3/strings/strings2.cc}
\lstinputlisting[caption = strings3.cc]{src/a3/strings/strings3.cc}

\end{document}
