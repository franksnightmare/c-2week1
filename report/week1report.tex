\documentclass[11pt]{article}

\usepackage{times}
\usepackage[english]{babel}

% -----------------------------------------------
% especially use this for you code
% -----------------------------------------------

\usepackage{courier}
\usepackage{listings}
\usepackage{color}
\usepackage{tabularx}
\usepackage{graphicx}

\definecolor{Gray}{gray}{0.95}

\definecolor{mygreen}{rgb}{0,0.6,0}
\definecolor{mygray}{rgb}{0.5,0.5,0.5}
\definecolor{mymauve}{rgb}{0.58,0,0.82}

\lstset{language=C++,
	basicstyle = \normalsize\ttfamily,   % the size and fonts that are used
	tabsize = 2,                    % sets default tabsize
	breaklines = true,              % sets automatic line breaking
	keywordstyle=\color{blue}\ttfamily,
	stringstyle=\color{red}\ttfamily,
	commentstyle=\color{mygreen}\ttfamily,
	numbers=left,
	keepspaces=true,
	showspaces=false,
	showstringspaces=false,
}

\begin{document}

\title{Programming in C/C++ \\
       Exercises set eight: Overloading
}
\date{\today}
\author{Christiaan Steenkist \\
Jaime Betancor Valado \\
Remco Bos \\
}

\maketitle

\section*{Exercise 1, catching and throwing references}
\subsection*{Exercise description}
There are 3 parts to this exercise:
\begin{itemize}
\item Show that exception catchers catching objects result in additional copies of thrown objects, compared to exception catchers catching references to objects.
\item Also show that when throwing objects or references copies of the (referred to) objects are thrown.
\item Also answer the question whether `throw;' results in throwing the currently available exception or a copy of that exception. 
\end{itemize}

\subsection*{1a, Throwing by value, catching by value}
Throwing object 'main object' by value.
Caught exception by value.
\begin{lstlisting}
// 2 copies are found
Hello by 'local object' (copy)  (copy)
\end{lstlisting}

\subsection*{1b, Throwing by value catching by reference}
Throwing object 'main object' by value.
Caught exception by reference.
\begin{lstlisting}
// 1 copy is found (answered part 2)
Hello by 'local object' (copy)
\end{lstlisting}

\subsection*{Part 2}
The '(copy)' is appended by the copy constructor, so atleast 1 copy is made
by throwing an object.

\subsection*{Part 3}
'Throw' throws the original exception. An exception is rethrown when it is not caught yet in the present try-block level, then the exception will be retrown to a higher level until it is caught.
That means that the exception is handled and will be inactivated.

\subsection*{Code listings}
\lstinputlisting[caption = demo.h]{src/a1/demo.h}
\lstinputlisting[caption = main.cc]{src/a1/main.cc}

\section*{Exercise 2, delete[]}
We designed a simple class that can only be constructed 4 times and then causes an exception.

\subsection*{Explanation}
First memory is allocated for 10 objects.
When the exception is reached during construction of objects, the construction of further objects is terminated.
This results in the extra allocated memory (i.e. objects 5 - 10 in the array) getting destroyed automatically.

\subsection*{Code listing}
\lstinputlisting[caption = maxfour.h]{src/a2/maxfour.h}
\lstinputlisting[caption = constructor.cc]{src/a2/constructor.cc}
\lstinputlisting[caption = destructor.cc]{src/a2/destructor.cc}
\lstinputlisting[caption = main.cc]{src/a2/main.cc}

\section*{Exercise 3, exceptions in the \texttt{Strings} class}
Exception handling has been put into the \texttt{Strings} class.
Generally bad allocations are handled by the class itself.
The constructor can still throw bad allocation exceptions in case there is not enough memory to create a strings class.

\subsection*{Code listing}
\lstinputlisting[caption = strings.h]{src/a3/strings/strings.h}
\lstinputlisting[caption = add.cc]{src/a3/strings/add.cc}
\lstinputlisting[caption = destroy2.cc]{src/a3/strings/destroy2.cc}
\lstinputlisting[caption = enlarged.cc]{src/a3/strings/enlarged.cc}
\lstinputlisting[caption = reserve.cc]{src/a3/strings/reserve.cc}
\lstinputlisting[caption = resize.cc]{src/a3/strings/resize.cc}
\lstinputlisting[caption = storagearea.cc]{src/a3/strings/storagearea.cc}
\lstinputlisting[caption = strings1.cc]{src/a3/strings/strings1.cc}
\lstinputlisting[caption = strings2.cc]{src/a3/strings/strings2.cc}
\lstinputlisting[caption = strings3.cc]{src/a3/strings/strings3.cc}

\section*{Exercise 4, calling it quits}
We had to find a way to quit no matter what but still call all the destructors.
To do this we made a very small custom class that is thrown so that no catch can intercept it.

\subsection*{Code listing}
\lstinputlisting[caption = example.cc]{src/a4/a4example.cc}

\section*{Exercise 6, throws}
Here we have different throws that might come from exercise 6.

\begin{tabular}{|c|c|c|}
\hline
No. & Statement & Reason \\ \hline
1. & throw x; & A copy of object x is thrown \\ \hline
2. & throw *xp; & The pointer *xp is thrown \\ \hline
3. & throw new X(x); & The pointer new X(x) is thrown \\ \hline
4. & throw X(x); & A copy of object X(x) is thrown \\ \hline
5. & throw (x + *xp); & A copy of (x + *xp) is thrown \\ \hline
6. & throw xp; & The address of xp is thrown \\ \hline
7. & throw; & An exception is thrown \\ \hline
8. & void fun() throw (type) & Only a type-exception is allowed \\ \hline
\end{tabular}

\end{document}
